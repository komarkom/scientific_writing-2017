\documentclass[conference]{IEEEtran}

\usepackage[utf8]{inputenc}
\usepackage[T2A]{fontenc}
\usepackage[russian,english]{babel}

\renewcommand{\sfdefault}{cmss}
\renewcommand{\rmdefault}{cmr}
\renewcommand{\ttdefault}{cmt}

\usepackage{booktabs}
\usepackage{csquotes}

\usepackage[%
style=ieee,
url=true,
defernumbers=true,
sorting=none,
bibencoding=utf8,
backend=biber,
language=auto,    % get main language from babel
autolang=other,
]{biblatex}
\addbibresource{refs.bib}

\begin{document}
	\title{Проблематика производительности многоагентных фреймворков для разработки ПО автономных роботов}
	\author{
		\IEEEauthorblockN{Михаил Ефремов}
		\IEEEauthorblockA{
			Санкт-Петербургский \\
			Электротехнический Университет\\
			СПбГЭТУ "ЛЭТИ" \\
			Email: jakutenshi@gmail.com}
	}

	\maketitle
	\selectlanguage{russian}
	\begin{abstract}
		Данная работа выделяет ряд проблем и задач, выполняемых таким промежуточным ПО, как фреймворки для разработки робототехнических приложений для автономных роботов. Данная статья делает упор на обнаружение критических областей для производительности в многоагентной архитектуре, так как подобный подход является наиболее пригодным для разработки рассматриваемых робототехнических фреймворков. Так же были выделены для исследования ряд существующих и наиболее используемых  фреймворков для рассмотрения способов решения проблем разработки такой архитектуры, выделяя подходы и технологии, которые необходимо протестировать для получения различных характеристик производительности распределенной системы.
	\end{abstract}

	\section{Введение}
		Объектом исследования данной статья являются многоагентные фреймворки для разработки ПО для автономных роботов (далее "фреймворки") на предмет критичных для производительности аспектов многоагентных робототехнических фреймворков.

Под фреймворками понимается набор промежуточного (middleware) ПО, находящегося по уровню абстракции между ОС и прикладными приложениями, предназначенного для управления неоднородностью аппаратного обеспечения с целью упрощения и снижения стоимости разработки ПО.
Кроме того, в состав фреймворков идет набор библиотек и, опционально, инструментов, для разработки прикладных программ, обслуживающих систему, в данном случае - автономного робота.

На рынке доступны различные версии фреймворков для автономных роботов с разными принципами работы, написанные на разных языках программирования и под разные платформы. В связи с тем, что появляются новые разработки, возникают новые задачи для автономных роботов, а так же технологии для разработки ПО для них - возникает желание рассмотреть доступные и развивающиеся в данный момент решения и проанализировать на предмет производительности, чтобы разработчики могли обосновывать свой выбор при разработки приложений для автономных роботов. При этом стоит учитывать как соответствие фреймворков возможным общим критериям (лицензия, статус разработки), так и важные для конкретной области: разработки ПО для роботов. Для выполнения тестирования, следует определиться с тем, какие задачи, выполняемые фреймворком, являются значимыми для производительности системы в целом.

Целью данной статью является получения списка проблем и задач, решаемых при разработке фреймворков, которые являются критичными для производительности робототехнической системы в целом.
	\section{Обзор предметной области }
		\input{tex/domain_overview.tex}
	\section{Выбор метода решения}
		\input{tex/solution_method_selection.tex}
	\section{Описание метода решения}
		\input{tex/solution_method_description.tex}
	\section{Заключение}
		В ходе исследования были отобраны 5 фреймворков по различным критериям, среди которых самые важные: открытый исходный код, понятная и доступная документация, децентрализованная (возможно, гибридная) архитектура, наличие инструментов мониторинга.

В ходе исследования многоагентной архитектуры робототехнических фреймворков были выявлены 6 основных источника влияния на производительность данного промежуточного слоя системы:

\begin{itemize}
	\item Наличие в системе централизованных узлов.
	\item Типы взаимодействия между узлами системы.
	\item Механизмы коммуницирования между узлами системы.
	\item Дополнительный функционал в системе коммуницирования, например шифрование или QoS.
	\item Типы сообщений, используемых для передачи информации между узлами системы.
	\item Реализация интерфейса доступа к аппаратной части автономного робота
\end{itemize}


Исходя из выявленных проблем была составлена таблица решений этих задач отобранными фреймворками, а так же сформулированы основные показатели производительности системы в зависимости от контекста рассматриваемой проблемы.

В данной работе и не рассматривалось некоторое количество фреймворков из-за централизованной архитектуры. В теории, опущенные из изучения разработки могут иметь интерес для тестирования производительности, но из-за разницы архитектур имеется большая разница в возникающих проблемах и задачах. Кроме того было показано, что сильно централизованные архитектуры плохо применимы к разработке робототехнических приложений для автономных роботов.

В дальнейшем, на основе полученной информации, планируется провести тестирование отобранных фреймворков в контексте выявленных узких мест для многоагентных робототехнических фреймворков.
		
	\printbibliography

\end{document}