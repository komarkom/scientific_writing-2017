Объектом исследования данной статья являются многоагентные фреймворки для разработки ПО для автономных роботов (далее "фреймворки") на предмет критичных для производительности аспектов многоагентных робототехнических фреймворков.

Под фреймворками понимается набор промежуточного (middleware) ПО, находящегося по уровню абстракции между ОС и прикладными приложениями, предназначенного для управления неоднородностью аппаратного обеспечения с целью упрощения и снижения стоимости разработки ПО.
Кроме того, в состав фреймворков идет набор библиотек и, опционально, инструментов, для разработки прикладных программ, обслуживающих систему, в данном случае - автономного робота.

На рынке доступны различные версии фреймворков для автономных роботов с разными принципами работы, написанные на разных языках программирования и под разные платформы. В связи с тем, что появляются новые разработки, возникают новые задачи для автономных роботов, а так же технологии для разработки ПО для них - возникает желание рассмотреть доступные и развивающиеся в данный момент решения и проанализировать на предмет производительности, чтобы разработчики могли обосновывать свой выбор при разработки приложений для автономных роботов. При этом стоит учитывать как соответствие фреймворков возможным общим критериям (лицензия, статус разработки), так и важные для конкретной области: разработки ПО для роботов. Для выполнения тестирования, следует определиться с тем, какие задачи, выполняемые фреймворком, являются значимыми для производительности системы в целом.

Целью данной статью является получения списка проблем и задач, решаемых при разработке фреймворков, которые являются критичными для производительности робототехнической системы в целом.